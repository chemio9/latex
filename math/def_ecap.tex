 \chapter{电容定义式}

 首先所谓的\emph{高斯定理}:
 \[
  \oint_{\mathcal S} \vec{E}(\vec{r})\cdot \dif \vec{s} = \frac{Q}{\epsilon_0} = \frac{1}{\epsilon_0} \int_{\mathcal S} \rho(\vec{r}) \dif V ~
 .\]

 看不懂,没关系。简单来讲就是我们在空间上画一个\emph{闭合曲面}(例如我们取个球面),
 如果一个点电荷在这个球外面,那么我们都知道有的电场线如果穿进这个球,那么它们也必定会穿出这个球,
 所以穿进去和穿出去的电场线必定相等的,这个说法不太严谨。
 我们学磁学时候学过法拉第电磁感应定律里面有个概念叫\emph{磁通量},这里也一样,就是电场强度 $E$ 和面积 $S$的乘积(虽然实际是积分,但是我们都可以意会)。$ES$ 就是电通量(磁通量 $\phi=BS$,是不是很像)。也就是说如果这个球里面没有点电荷,那么净通量就是 $0$(进去的等于出去的)。而如果球里面有正电荷呢,那出去的就应该比进去的多。也就是说我们得到一个重要结论:

 对于封闭曲面,净通量=源的强度。

 高斯定理定性分析其实就是这么一回事。特殊的,如果我们有一个点电荷 $Q$ 位于一个球的球心位置,那么球面上由于对称电场强度大小一样。所以 $E$ 不变,那么净通量就是 $ES$,那源强度应该跟电荷量 $q$ 成正比。那么具体公式是什么呢,答案是

 $$
 \vec{E}S=\frac{Q}{\epsilon_0}
 $$

 可能有的小朋友对这个公式很陌生,那么我们不妨推导一下,球的面积

 $$
 S=4\pi r^2
 $$

 $$
 \vec{E}=\frac{Q}{4\pi r^2 \epsilon_0}
 $$

 我们令 $k=\frac{1}{4\pi \epsilon_0}$,因为这玩意都是常数。然后得到

 $$
 \vec{E}=k\frac{Q}{r^2}
 $$

 我们给 $k$ 取了个名字叫库伦常数,是不是突然想起来这是什么了,这不就是\emph{点电荷电场强度公式}吗。所以高斯定理讲的就是电通量和电荷量的比例关系。

 $$
 \vec{E} S=\frac{Q}{\epsilon_0}
 $$

 这个 $\epsilon_0$ 叫真空介电常数,物理含义很明显吧,就是个比例系数。那么回到电容这里,我们知道电容定义 $C=\frac QU$。也就是说如果我能用 $Q$ 表示出来 $U$ 多大是不是就解决了。我们刚才说了 $\vec{E}S=\frac{Q}{\epsilon_0}$,那我们还知道 $\vec{E} \cdot \vec{d}=U$ ,所以

 $$
 U=\vec{E}\cdot d =\frac{Q}{S\cdot \epsilon_0} \cdot \vec{d}
 $$

 $$
 C=\frac{Q}{U}=\frac{\vec{E}S\cdot \epsilon_0}{Ed}=\frac{S\cdot\epsilon_0}{d}=\frac{S}{4\pi kd}
 $$

 我们发现现在跟书上给的就差了一个 $\epsilon_r$,这又是什么玩意呢,原理是我们刚才用到的 $\epsilon_0$ 那个是真空条件下的。如果你中间介质不是真空的还要再多乘一个比例系数,我们刚才算的那个 $C$ 是 $C_0$,实际 $C=\epsilon_r\cdot C_0$。也就是

 $$
 C=\frac{\epsilon_r \cdot S}{4\pi k}
 $$
