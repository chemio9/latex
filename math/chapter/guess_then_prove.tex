\chapter{先猜后证}

先猜后证可理解为必要性探路,该思维上可用于解决世界数学难题,下可解决高中数学问题。

在牛顿之前的时期,人们热衷于求极限,这极大促进了微积分的发展。
在这样的背景下,1650 年 (此时欧拉 57 岁,牛顿玩泥巴的年纪),
数学家门戈利开始研究所有自然平方后的倒数之和
\[
  \sum_{n=1}^{\infty} n^{-1} = \frac{1}{1^2} + \frac{1}{2^2} +
  \frac{1}{3^2} + \ldots
.\]

由于当时工具有限,他只好硬算。
在经过大量的计算后他发现,在$n=998$时,其和为$1.643932$,在$n=999$时,其值为$1.643933$,
在$n=1000$时,其值为$1.643934$
(这其中的工作量可想而知)。
他推测在$n$无限大时,其和为一个定值,这就是赫赫有名的“巴塞尔问题”。
这个问题经过一众大神的轮番研究后仍未解决。

直到 1735 年,数学之王欧拉接触到了巴塞尔问题,在他看到门戈利算出的$1.643934$时,他想都没想,
直接说“这不就是的$\frac{\pi^2}{6}$吗.”(这直觉太恐怖了😱天赋怪)
之后他便按照他就沿着他认为的这个答案的方向探索,经过一系列复杂的计算,
将巴塞尔问题解决了,而最终的结果确实就是$\frac{\pi^2}{6}$。

实际上,这就是先猜后证的一个典例了。
看着答案来做证明,过程可是简单多了。
在平时做题中,适当应用这个方法,会有意想不到的结果。
当然,我们也应当学习门戈利的毅力,遇到计算不要害怕。
正是他的计算,帮助欧拉一眼看出问题的答案。
