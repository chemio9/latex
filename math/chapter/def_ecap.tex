\chapter{电容定义式}

首先所谓的\emph{高斯定理}:
\[
  \oint_{\mathcal S} \vec{E}(\vec{r})\cdot \dif \vec{s} =
  \frac{Q}{\epsilon_0} = \frac{1}{\epsilon_0} \int_{\mathcal S}
  \rho(\vec{r}) \dif V ~
.\]

看不懂,没关系。简单来讲就是我们在空间上画一个\emph{闭合曲面}(例如我们取个球面),
如果一个点电荷在这个球外面,那么我们都知道有的电场线如果穿进这个球,那么它们也必定会穿出这个球,
所以穿进去和穿出去的电场线必定相等的,这个说法不太严谨。
我们学磁学时候学过法拉第电磁感应定律里面有个概念叫\emph{磁通量},这里也一样,就是电场强度 $E$ 和面积
$S$的乘积(虽然实际是积分,但是我们都可以意会)。$ES$ 就是电通量(磁通量
$\phi=BS$,是不是很像)。也就是说如果这个球里面没有点电荷,那么净通量就是
$0$(进去的等于出去的)。而如果球里面有正电荷呢,那出去的就应该比进去的多。也就是说我们得到一个重要结论:

对于封闭曲面,净通量=源的强度。

高斯定理定性分析其实就是这么一回事。特殊的,如果我们有一个点电荷 $Q$
位于一个球的球心位置,那么球面上由于对称,电场强度处处大小相等。所以 $E$ 不变,那么净通量就是 $ES$,那源强度应该跟电荷量 $q$
成正比。那么具体公式是什么呢,答案是

$$
ES=\frac{Q}{\epsilon_0}
$$

可能有的同学对这个公式很陌生,那么我们不妨推导一下,球的面积

$$
S=4\pi r^2
$$

$$
E=\frac{Q}{4\pi r^2 \epsilon_0}
$$

我们令 $k=\frac{1}{4\pi \epsilon_0}$,因为这玩意都是常数。然后得到

$$
E=k\frac{Q}{r^2}
$$

我们给 $k$ 取了个名字叫库伦常数,是不是突然想起来这是什么了,这不就是\emph{点电荷电场强度公式}吗。所以高斯定理讲的就是电通量和电荷量的比例关系。

$$
ES=\frac{Q}{\epsilon_0}
$$

这个 $\epsilon_0$ 叫真空介电常数,物理含义很明显吧,就是个比例系数。
那么回到电容这里,我们知道电容定义 $C=\frac QU$。
也就是说如果我能用 $Q$ 表示出来 $U$ 多大是不是就解决了。
刚才 $ES=\frac{Q}{\epsilon_0}$,我们还知道 $Ed=U$ ,所以

$$
U=Ed =\frac{Q}{S\cdot \epsilon_0} \cdot d
$$

$$
C=\frac{Q}{U}=\frac{ES\cdot
\epsilon_0}{Ed}=\frac{S\cdot\epsilon_0}{d}=\frac{S}{4\pi kd}
$$

我们发现现在跟书上给的就差了一个 $\epsilon_r$,这又是什么玩意呢,原理是我们刚才用到的 $\epsilon_0$
那个是真空条件下的。如果你中间介质不是真空的还要再多乘一个比例系数$\epsilon_r$,我们刚才算的那个 $C$ 是 $C_0$,实际
$C=\epsilon_r\cdot C_0$。也就是

$$
C=\frac{\epsilon_r \cdot S}{4\pi k}
$$
