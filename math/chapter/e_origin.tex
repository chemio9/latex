\chapter{``$\mathrm{e}$''的起源}

1727 年,欧拉刚满 20 岁,这时候他研究生毕业,找不到工作
(是的,在当时高斯、黎曼、爱因斯坦等人也找不到工作).
不过欧拉的好朋友 丹尼尔\textperiodcentered 伯努利 在俄罗斯圣彼得堡科学院当教导主任,
于是欧拉就去医学院当生理卫生老师(魏尔斯特拉斯 都当过体育老师,所以从源头上说,我们的数学不是生理卫生老师教的就是体育老师教的🤣)
学习了两个月生理卫生课后他就跑去圣彼得堡了。
他因有工作而开心,但没有因此而忘记使他开心的数学。
他经常在下班后搞点数学玩玩,从指数函数$y=2^{x}$,
和导数的定义开始爆算。\footnote{注:当时未有极限的概念.}

\begin{align*}
  \left( 2^{x}  \right)' & = \lim_{h \to 0} \frac{2^{x+h}-2^{x} }{h} \\
  & = \lim_{h \to 0} \frac{2^{x}\cdot 2^{h}-2^{x}   }{h} \\
  & = 2^{x} \lim_{h \to 0} \frac{2^{h}-1 }{h}
\end{align*}

这样,$2^{x} $的导数变成它自己乘以一个系数。

这个系数是多少呢,欧拉硬算一下发现它约等于$0.6931$,
那$y=3^x$呢,他算了一下发现仍是一个系数约等于$1.09861$,
如果继续算下去就会发现$4$、$5$、$6$、$7$的$x$次方对应的系数
依次为$1.3862$,$1.6094$,$1.7917$,$1.9459$,
到这里欧拉发现一个有趣的问题:
我们应以怎样一个数为底其对应系数为$1$呢,换句话说,
``会不会有一个指数函数的导数等于它本身呢'',欧拉心想,那算吧:

我们先假设有这么个数$a$

\begin{align*}
  \left(a^{x}\right)' & = \lim_{h \to 0} \frac{a^{x+h}-a^{x}}{h}  \\
  & = \lim_{h \to 0} a^{x}\cdot a^{h} - a^{x} \\
  & = a^{x} \lim_{h \to 0} \frac{a^{h}-1}{h}
\end{align*}
令这个系数等于 1.

\begin{align*}
  \lim_{h \to 0} \frac{a^{h}-1}{h} &= 1 \\
  \lim_{h \to 0} a^{h} - 1 &=\lim_{h \to 0} h \\
  \lim_{h \to 0} a^{h} &=\lim_{h \to 0} h+1 \\
  a &=\lim_{h \to 0} (h+1)^{\frac{1}{h}}
  \intertext{换个元,我们就有}
  a &=\lim_{n \to \infty} (1+\frac{1}{n})^{n}
\end{align*}

于是欧拉爆算发现这个值约等于$2.71828182845904(5)$,
其实这个数在莱布尼茨时期已被发现,
后人将其称为欧拉数,即``$\upe$''.

在当时,像$\frac{\pi}{4}$这类无理数的无穷级数形式表示是热门话题,
那么\upe 的无穷级数形式是什么样的呢?

我们利用一下刚才的结论
$(\upe ^{x})' = \upe ^{x} $, 我们可以猜测一下它的无穷级数:
\begin{equation}
  e^{x}= 1+x + \frac{x^2 }{2!} + \frac{x^3}{3!} + \ldots +
  \frac{x^{n} }{n!} \label{eqn:ex}
\end{equation}

因为我们知道幂函数的求导公式$(x^{n})' = nx^{n-1}  $,
对上式求导得
\begin{align*}
  (e^{x})' &= 1' &+ x' &+ \left(\frac{x^2}{2!}\right) &+
  \left(\frac{x^3}{3!}\right)' &+ \ldots &+
  \left(\frac{x^{n}}{n!}\right)' &\quad \\
  &= 0 &+ 1 &+ x
  &+ \frac{x^2}{2!} &+ \ldots &+ \frac{x^{n-1}}{(n-1)!} &+\ldots
\end{align*}

式子整体不变,恰好满足了 $\left(\upe^{x}\right)' = \upe^{x}  $
于是乎这就是$e^{x}$的无穷级数。

带入$x=1$,得
\[
  \upe = \frac{1}{0!} + \frac{1}{1!} + \frac{1}{2!} + \ldots
.\]

(是的,这是欧拉凭空想象得来的,可见想象力多么重要🤓)
\Cref{eqn:ex} 实际上就是$\upe ^{x} $在$x=0$处的泰勒展开。
