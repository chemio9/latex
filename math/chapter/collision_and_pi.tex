\chapter{碰撞的物块与圆周率:一场奇妙的物理之旅}

物体的碰撞和圆周率,这两者看起来似乎毫无关系,但在一次巧合之中,科学家们发现了他们之间的关联,令人惊讶。

假定在一个无摩擦的平面上有 2
个小物块,左侧的方块初始$v_{2}=0$,质量$m_{2}$小于等于右侧物块$m_{1}$,右侧物块则以一定的速度$v_{1}$向左运动,两个方块会互相碰撞,设为弹性碰撞,他们左边有一堵墙。
那么在这么一个系统之内,他们总共会发生多少次碰撞(包括左侧物块与墙壁的碰撞)呢?

有一个特别重要的思想,就是从特殊到一般。

让我们先假定两个物块质量$m_{1}=m_{2}=\SI{1}{\kg}$.
那么很明显,由于他们之间发生了弹性碰撞,且无内能的转化,
由物理必修课本中的机械能守恒有
\[
  \frac{1}{2} m_{1}v_{1}^2+\frac{1}{2} m_{2}v_{2}^2  =
  \frac{1}{2} m_{1}v_{1}'^2 + \frac{1}{2} m_{2}v_{2}'^2
.\]

又由选修一课本中的动量定理
\[
  m_{1}v_{1}+m_{2}v_{2}=
  m_{1}v_{1}' + m_{2}v_{2}'
.\]

我们可知,
\begin{align*}
  v_{1}' = \frac{(m_{1}-m_{2})v_{1} + 2m_{2}v_{2}}{m_{1}+m_{2}} \\
  v_{2}' = \frac{(m_{2}-m_{1})v_{2} + 2m_{1}v_{1}}{m_{1}+m_{2}}
\end{align*}

通过简单的物理计算就可以知道,其总共会碰撞 3 次。

倘若我们把右边的物块质量增大到$m_{1}=\SI{100}{\kg}$, 通过电脑的模拟计算,我们也可以知道他们总共碰撞了 31 次。

再增大一点到 \SI{10000}{\kg}, 计算得知他们总共会碰撞 314 次。
看着这些碰撞的次数,你想起了什么吗?$314$, $3.14$ ,$\pi$?
这会是一个偶然的巧合吗?那再让我们继续增大质量,增大到\SI{1000000}{\kg}, 计算机屏幕上的小物块经过多次碰撞之后,数字定格在了 3141 次。
好吧,看起来这好像并不是一个巧合,他碰撞的次数似乎与$\pi$之间有一种微妙的关系,接下来就应该一般分析了。

让我们设右侧物块的质量为$m_{1}$,左侧物块的质量为$m_{2}$.
因为能量守恒,所以我们可以写出下式:

\[
  E = \frac{1}{2} m_{1}v_{1}^2 + \frac{1}{2} m_{2}v_{2}^2
.\]

在碰撞过程中,$m_{1},m_{2},E$不变,变量有$v_{1},v_{2}$.
在数据分析的时候,借助数形结合来解答问题是较为常用的一种方法。
上式中的两个变量,我们就可以这样处理。
我们设一个点为$(x,y)$,以$v_{1}$为$x$轴,$v_{2}$为$y$轴,建立平面直角坐标系。
不妨设右侧为速度的正方向。
那么上式的形式就可以看做$ax^2+by^2=c$.
我们可以发现这一条表达式所对应的图形是椭圆\footnote{详见数学课本选择性必修一圆锥曲线一章}.

%%

但是我们的高中知识的确有限,对于椭圆的研究比较少,因此,我们无法对椭圆进行直接的研究。
为了顺利的研究这个图形,我们就可以采用数学的另一种高级方法,叫做映射。
\footnote{ 映射 (map) 是函数概念的推广,它描述了两个集合之间的关系,如果这两个集合是同一个,那么映射也称作变换。}
通过化椭圆为圆,从而将我们所学过的知识关联起来,以顺利解决这个问题。

让我们通过线性映射(即仿射变换)重新将$\sqrt{m_{1}}v_{1} $设为$x$轴,将$\sqrt{m_{2}}v_{2}
$设为$y$轴,那么我们就可以得到一个美妙的式子:$x^2+y^2=E$.
很成功地,我们将椭圆转化成了圆,接下来我们在数学工具中画出它的图形:

%%

因为$x^2$和$y^2$的和不变,
无论两个物块的初始动能为何值,它们所对应的点$\left(\sqrt{m_{1}}v_{1}, \sqrt{m_{2}}v_{2}
\right)$都在该圆上,和为$2E$. 碰撞过程可视为点在图上运动的过程。

接下来我们来考虑动量设总动量为$P$,则有:
\[
  m_{1}v_{1}+m_{2}v_{2} = P
.\]

根据上述坐标系,我们可以将上式写成:
\[
  \sqrt{m_{1}}\left( \sqrt{m_{1}}v_{1} \right) + \sqrt{m_{2}} \left(
  \sqrt{m_{2}}v_{2} \right) = P
.\]

可以看出来,这条式子所表示的是一条直线$\sqrt{m_{1}} x + \sqrt{m_{2}}y = P  $.
由于$v_{2}$初始值为$0$,所以在碰撞前,它必处于上图红点处,而且该直线的斜率为$-\sqrt{\frac{m_{1}}{m_{2}}} $.

(这里由于$m_{1},m_{2}$为变量,所以我们先随便画一条$k<0$的线)
过红点做斜率为$-\sqrt{\frac{m_{1}}{m_{2}}} $的一条直线交于圆于第二个点,该点就为第一次双物块碰撞后所处于的状态点。

让我们来分析左侧物块与墙壁的碰撞,由于该系统总动能未变,所以碰撞后点必然在圆上。
由于$v_{2}$不改变,$v_{1}$反向,所以我们只需要作出过第二个点垂直于 x
轴的一条直线,它与圆的另一个交点就是左侧物块与墙壁碰撞后,系统所处于状态点。
以此类推,我们便可以做出下图:

那么,这样的碰撞什么时候是个头呢?
我们来假设一下,若左侧物块碰撞后速度向左,那么它最后一定会与墙壁再发生碰撞,所以左侧物块最终碰撞后的速度应向右。
同理,右侧的物块速度也应向右。
若左侧物块向右的最终速度大于右侧物块向右的速度,则两个物块还会发生一次碰撞。
所以左侧物块最终末速度应当小于右侧物块的末速度,即$0< v_{2} < v_{1}$, 所以该点应当位于第一象限,
且在$v_{1}=v_{2}$所在的直线$y=\sqrt{\frac{m_{2}}{m_{1}}}x$与$x$轴围成的区域内,

我们在图中做出这条直线:

所以当点位于上图绿色区域内时 (包含边界),该系统将不会再发生碰撞,他们之间碰撞的次数可以通过数上图的线段的多少的得知。
到此,我们就把这个物理问题转化成了一个数学问题。

我们就上图进行分析,设每一条倾斜的直线的倾斜角为$\theta $,则
$\tan \theta  = -\sqrt{\frac{m_{1}}{m_{2}}} $ ,
看着上面那一个图,你会不会觉得是很规则的锯齿状?我们稍微研究一下他们的几何关系,容易发现圆上的每一段弧所对应的圆周角都是相等的
(瞪眼,平行秒啦),再而推出每一段的弧长与圆心角也是相等的。

由于绿色区域上边缘所在的直线的斜率为$\sqrt{\frac{m_{2}}{m_{1}}} $,所以其对应的倾斜角就为$\theta $.
如图不难看出,最右端的这段弧长所对应的圆心角小于等于$\theta $.
所以实际上我们就把题目上的问题转化成了:这一个圆上可以存在多少段圆心角为$2\theta$的弧
(由图像其实不难看出,圆内弦的条数与其所拥有的弧数其实是相等的,故此处不再说明)?
设在这圆上有$n$段这样的弧,那我们就可以得出来:$n \cdot 2 \theta \cdot R \le 2 \pi R$.

经过化简,我们就可以得到: $n \cdot \theta \le  \pi $,
\footnote{ 高一的新同学,请注意这用的$\theta$为弧度制单位下的角度表达,并不是在初中时常用的角度制 }
在前文中,我们猜想是$n$与$\pi $有一定的关系。
然而根据这个式子,我们并不能很好地看出来它们之间有什么样的直接关系。
让我们回到前文可以看出来,随着右侧物块的质量与左侧物块质量之比逐渐增加时,
$n$似乎在逐渐靠近$\pi $的各位数字,我们不妨把这个比值再次增大。
这时候,$\theta $的值其实在不断的减小 ($y=\tan x$ 在 $\left( - \frac{\pi}{2},0
\right) $ 上是一个增函数).

当这个比值足够大的时候,那么$\tan
\theta$的值也会逐渐变小,我们又由小角近似可知,在$\theta$足够小的时候,$\theta \approx \tan \theta $.
于是我们上面的式子可以改写成:$n \cdot \tan \theta \le  \pi $.
这个$\tan \theta$可以是多小呢?
我们假设它小到可以忽略其具体数值,只保留数量级,那么上式就可以写成:$n \cdot 10^{-q} \le  \pi  $
由于$n$是正整数,所以该式就可以理解为$q$位精确度下$\pi $的数值。

到此,他们之间关系就被我们发现了。

经过科学家们的验证,在现有计算仪器的精度下,这样子计算出来的$\pi $的近似值是较为精确的。
所以我们的证明过程也差不多结束了。

纯粹的科学家们善于从现实世界的混乱中将核心抽离出来,更深层次的原因是纯粹可以暴露隐藏的联系。
之前,格雷戈里·盖普 (Gregory Gap)
的论文,一束光在两面镜子之间以一定角度反射,最终光束反射的次数也与$\pi $有密切的关系,
该现象与数学之间的微妙关系的确令人惊叹。

在最后笔者想要告诉各位读者的是:科学的魅力在于它能够揭示看似无关的现象之间的深刻联系。
物体的碰撞次数与圆周率$\pi $之间的关系,就是一个典型的例子,展示了生活中隐藏的数学之美。
这种美不仅仅是形式上的,更是功能上的,它体现了宇宙运行的基本规律。
在探索这个现象的过程中,我们运用了物理学、数学一些相关知识。
这些知识帮助我们将复杂的问题简化,从而揭示问题的本质。
这也提醒我们,我们的学习不仅仅是记忆公式和事实,更是一种思维方式,一种解决问题的方法。
对于我们每个人来说,这些知识教会我们如何观察世界,如何思考问题,如何寻找答案。
在这个信息爆炸的时代,我们需要的不仅仅是知识,更是一种能够帮助我们理解世界的能力。
让我们保持好奇心,保持求知欲,不断探索,不断发现。
就像那些伟大的科学家一样,从看似平凡的现象中发现不平凡的真理。
只有这样,我们才能真正理解宇宙所蕴含的奥秘,才能真正享受到科学带来的乐趣。
