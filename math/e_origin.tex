\documentclass[UTF8]{mathrep}

\newcommand*{\upe}{\ensuremath{\mathrm{e}}}
\usepackage{emotion}
\ExplSyntaxOn
\newfontfamily \emotionfont:
{/usr/share/fonts/apple-color-emoji/apple-color-emoji.ttf}
\ExplSyntaxOff

\title{"$\mathrm{e}$"的起源}
\date{\today}
\author{雅思数论}

\begin{document}

\maketitle

1727年,欧拉刚满20岁,这时候他研究生毕业,找不到工作
(是的,在当时高斯、黎曼、爱因斯坦等人也找不到工作).
不过欧拉的好朋友 丹尼尔•伯努利 在俄罗斯圣彼得堡科学院)当教导主任,
于是欧拉就去医学院当生理卫生老师(魏尔斯特拉斯都当过体育老师,所以从源头上说,我们的数学不是生理卫生老师教的就是体育老师教的\emotion{🤣})
学习了两个月生理卫生课后他就跑去圣彼得堡了。
他因有工作而开心,但没有因此而忘记使他开心的数学.
他经常在下班后搞点数学玩玩,从指数函数$y=2^{x}$开始,
从导数的定义开始爆算.\footnote{注:当时未有极限的概念.}

\begin{align*}
  \left( 2^{x}  \right)' & = \lim_{h \to 0} \frac{2^{x+h}-2^{x}  }{h}
  \\
  & = \lim_{h \to 0} \frac{2^{x}\cdot 2^{h}-2^{x}   }{h} \\
  & = 2^{x} \lim_{h \to 0} \frac{2^{h}-1 }{h}
\end{align*}

这样,$2^{x} $的导数变成它自己乘以一个系数.

这个系数是多少呢,欧拉硬算一下发现它约等于$0.6931$,
那$y=3^x$呢,他算了一下发现仍是一个系数约等于$1.09861$,
如果继续算下去就会发现$4$、$5$、$6$、$7$的$x$次方对应的系数
依次为$1.3862$,$1.6094$,$1.7917$,$1.9459$,
到这里欧拉发现一个有趣的问题:
我们应以怎样一个数为底其对应系数为$1$呢,换句话说,
会不会有一个指数函数的导数等于它本身呢,欧拉心想,那算吧:
我们先假设有这么个数a

\begin{align*}
  \left(a^{x}\right)' & = \lim_{h \to 0} \frac{a^{x+h}-a^{x}}{h}  \\
  & = \lim_{h \to 0} a^{x}\cdot a^{h} - a^{x} \\
  & = a^{x} \lim_{h \to 0} \frac{a^{h}-1}{h}
\end{align*}
令它等于0.

\begin{align*}
  \lim_{h \to 0} \frac{a^{h}-1}{h}
  % TODO
\end{align*}

于是欧拉爆算发现这个值约等于$2.718281828459045$,
其实这个数在莱布尼茨时期已被发现,
后人将其称为欧拉数,即“$\upe$”,在当时直到现在,
像$\frac{\pi}{4}$这类无穷级数是十分热门的话题,
于是我们利用一下刚才的结论,
$(\upe ^{x})' = \upe ^{x}  $ 我们可以猜测一下它的无穷级数:
因为我们知道幂函数的求导公式【见图三】

于是乎这就是$e^{x}$的无穷级数.
(是的,这是欧拉凭空想象得来的,可见想象力多么重要\emotion{🤓})
这实际上就是$\upe ^{x} $在$x=0$处的泰勒展开.

\end{document}
